\documentclass[12pt]{article}

\usepackage[utf8]{inputenc}
\usepackage[T1]{fontenc}
\usepackage[catalan]{babel}
\usepackage{lmodern}
\usepackage{geometry}
\usepackage{hyperref}
\usepackage{xcolor}
\usepackage[bf,sf,small,pagestyles]{titlesec}
\usepackage[font={footnotesize, sf}, labelfont=bf]{caption} 
\usepackage{siunitx}
\usepackage{graphicx}
\usepackage{amsmath,amssymb}
\usepackage[catalan]{cleveref}

\hypersetup{
	colorlinks,
	linkcolor = {red!50!blue},
	linktoc = page
}

\geometry{
	a4paper,
	right = 2.5cm,
	left = 2.5cm,
	bottom = 3cm,
	top = 3cm
}

\crefname{figure}{figura}{figures}

\graphicspath{{./figs/}}

% Unitats
\sisetup{
	inter-unit-product = \ensuremath{ \cdot },
	allow-number-unit-breaks = true,
	detect-family = true,
	list-final-separator = { i },
	list-units = single
}

% Format de títols i de pàgina
\newpagestyle{pagina}{
	\headrule
	\sethead*{}{}{\sffamily \sectiontitle}
	\footrule
	\setfoot*{}{}{\sffamily \thepage}
}
\renewpagestyle{plain}{
	\footrule
	\setfoot*{}{}{\sffamily \thepage}
}
\pagestyle{pagina}

\titleformat{\section}[hang]{\bfseries \sffamily \Large}{}{0pt}{}{\thispagestyle{plain}}

\newcommand{\Z}{\mathbb{Z}}
\newcommand{\N}{\mathbb{N}}
\newcommand{\R}{\mathbb{R}}
\newcommand{\abs}[1]{\left\lvert #1 \right\rvert}
\renewcommand{\vec}[1]{\mathbf{#1}}
\newcommand{\parbreak}{
	\begin{center}
		--- $\ast$ ---
	\end{center} 
}
\makeatletter
\newcommand*{\defeq}{\mathrel{\rlap{%
    \raisebox{0.3ex}{$\m@th\cdot$}}%
  \raisebox{-0.3ex}{$\m@th\cdot$}}%
=}
\makeatother

\title{\sffamily {\bfseries Entrega 3:} Conducció de la calor}
\author{\sffamily Arnau Mas}
\date{\sffamily 4 de Maig 2018}

\begin{document}
\maketitle

\section{Problema 25}
Considerem un tub cílindric de conductivitat constant \( \lambda \) que té un radi intern \( R_1 \) que es manté a temperatura constant \( T_1 \) i un radi extern que es manté a tempeartura constant \( T_2 \). Com que no hi ha fonts, l'equació de la temperatura és  
\begin{equation} \label{eq:eq de la temperatura}
	\dot{T} = \alpha \nabla^2 T,
\end{equation}
on \( \alpha \defeq \frac{\lambda}{\rho c_e} \) és la difusivitat del material ---\( \rho \) i \( c_e \) són, respectivament, la densitat i calor específica del material---. Si considerem l'estat estacionari, és a dir, quan \( \cdot{T} = 0 \) aleshores l'\cref{eq:eq de la temperatura} esdevé
\begin{equation} \label{eq:eq de la temperatura estacionaria}
	\nabla^2 T = 0. 
\end{equation}
Si prenem coordenades cílindriques \( (r, \theta, z) \) per explotar la simetria del problema podem escriure \( T = T(r) \) ja que, per simetria, la temperatura no pot dependre ni de l'angle \( \theta \) ni de la coordenada \( z \). Així doncs, en coordenades, l'\cref{eq:eq de la temperatura estacionaria} és
\begin{equation} \label{eq:eq de laplace coordenades}
	\frac{1}{r} \frac{\partial}{\partial r} \left( r \frac{\partial T}{\partial r} \right) = 0. 
\end{equation}
La solució general d'aquesta equació és \( T(r) = A\log{r} + B \), ja que aleshores es compleix que \( r \partial_rT \) és una constant, la qual cosa satisfa l'\cref{eq:eq de laplace coordenades}. Imposem les condicions de contorn per determinar \( A \) i \( B \):
\begin{align*}
	T(R_1) &= A\log{R_1} + B = T_1 \\
	T(R_2) &= A\log{R_2} + B = T_2 
\end{align*}
implica
\begin{equation*}
 	A = \frac{T_1 - T_2}{\log{R_1} - \log{R_2}} 
\end{equation*}
i per tant
\begin{equation*}
	B = T_1 - A\log{R_1} = \frac{T_2 \log{R_1} - T_1 \log{R_2}}{\log{R_1} - \log{R_2}}. 
\end{equation*}
Així doncs, la distribució de temperatures és
\begin{equation} \label{eq:distribucio de temperatura}
	\begin{aligned}
		T(r,\theta,z) & = \frac{T_1 - T_2}{\log{R_1} - \log{R_2}}\log{r} + \frac{T_2 \log{R_1} - T_1 \log{R_2}}{\log{R_1} - \log{R_2}} \\
									& = \frac{T_1 \log{(r/R_2)} - T_2 \log{(r/R_1)}}{\log{(R_1/R_2)}}.
	\end{aligned}
\end{equation}

Podem fer servir \cref{eq:distribucio de temperatura} i la llei de Fourier per determinar la densitat de flux de calor \( \vec{q} \):
\begin{equation*}
	\vec{q} = -\lambda \nabla T = -\lambda \frac{\partial T}{\partial r} \vec{e}_r =  \frac{\lambda \left(T_2 - T_1\right)}{\log{R_1} - \log{R_2}} \frac{\vec{e}_r}{r} .
\end{equation*}
Per tant, si considerem un cilindre \( \mathcal{C} \) centrat a l'eix del tub de radi \( R_1 \leq r \leq R_2 \) i longitud \( L \), la potència \( \dot{Q} \) que atravessa la seva superfície és
\begin{equation*}
	\dot{Q} = \int_{\mathcal{C}} \vec{q} \cdot d\vec{a} = \frac{\lambda \left(T_2 - T_1\right)}{\log{R_1} - \log{R_2}} \int_0^L \int_0^{2\pi} \frac{1}{r} r \, d\phi \, dz = \frac{2\pi \lambda L \left(T_2 - T_1\right)}{\log{R_1} - \log{R_2}} ,
\end{equation*}
ja que la densitat de flux és perpendicular a les tapes del cilindre. 

Per tant, la potència per unitat de longitud que atravessa la superfície d'un cilindre de radi \( R_1 \leq r \leq R_2 \) centrat a l'eix del tub és 
\begin{equation*}
	\frac{\dot{Q}}{L} = \frac{2\pi\lambda \left(T_2 - T_1\right)}{\log{R_1} - \log{R_2}}.
\end{equation*}


\end{document}
