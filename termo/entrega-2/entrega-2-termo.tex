\documentclass[12pt,a4paper]{article}
\usepackage{header}

\title{\textsf{\textbf{Entrega 2}: Cadenes de desintegració radioactiva}}
\author{\textsf{Arnau Mas}}
\date{\textsf{23 de Març de 2018}}

\begin{document}
\maketitle
Considerem la següent cadena de desintegracions
\begin{equation*}
	X \to Y \to Z
\end{equation*}
on Z és un nucli estable i \( X \) i \( Y \) tenen constants de desintegració \( \lambda_X \) i \( \lambda_Y \) respectivament. Suposem que tenim un nombre inicial \( N \) de nuclis.

Tenim que el nombre de nuclis \( X \), \( N_X \), decau com \( -A_X \), on \( A_X \) és l'activitat de \( X \) és a dir, el nombre de desintegracions de \( X \) en \( Y \). Per tant \( \dot{N}_X = -A_X \). El nombre de nuclis de \( Y \)   també decau com \( -A_Y \) on ara \( A_Y \) és l'activitat de \( Y \), és a dir, el nombre de desintegracions de \( X \) en \( Z \). Però, com que \( A_X \) era el nombre de nuclis de \( X \) que es desintegraven en \( Y \), \( N_Y \) també incrementa com \( A_Y \). Per tant \( \dot{N}_Y = A_X - A_Y \). Així doncs, tenint en compte que \( A_X = \lambda_X N_X \) i que \( A_Y = \lambda_Y N_Y \) tenim el següent sistema d'equacions diferencials:
\begin{equation} \label{eq:eqs diferencials}
	\left.
		\begin{aligned}
			\dot{N}_X &= -\lambda_X N_X \\
			\dot{N}_Y & = \lambda_X N_X - \lambda_Y N_Y
		\end{aligned}	
	\right\}.
\end{equation}
En forma matricial podem escriure
\begin{equation} \label{eq:eqs diferencials matriu}
	\begin{pmatrix}
		\dot{N}_X \\
		\dot{N}_Y 
	\end{pmatrix}
	=
	\begin{pmatrix}
		- \lambda_X & 0 \\
		\lambda_X & -\lambda_Y 
	\end{pmatrix}
	\begin{pmatrix}
		N_X \\
		N_Y
	\end{pmatrix}.
\end{equation}
La matriu d'aquest sistema té valors propis \( -\lambda_X \) i \( -\lambda_Y \), per tant diagonalitza. Per valor propi \( -\lambda_X \) el sistema es redueix a \( \dot{N}_X = -\lambda_X N_X \) i \( N_Y = \frac{\lambda_X}{\lambda_Y - \lambda_X} N_X \). I pel valor propi \( -\lambda_Y \) tenim \( \dot{N}_Y = -\lambda_Y N_Y \) i \( N_X = 0 \). Així doncs, la solució general del sistema és 
\begin{equation} \label{eq:sol general}
\left.
	\begin{aligned}
		N_X(t) & = Ae^{-\lambda_X t} \\
		N_Y(t) & = A\frac{\lambda_X}{\lambda_Y - \lambda_X}e^{-\lambda_X t} + Be^{-\lambda_Y t}
	\end{aligned}
\right\}.
\end{equation}

Si a \ref{eq:sol general} imposem condicions inicials \( N_X(0) = N \) i \( N_Y(0) = 0 \) trobem que \( A = N \) i \( B = -N\frac{\lambda_X}{\lambda_Y - \lambda_X} \) i per tant 
\begin{equation*}
	\left.
		\begin{aligned}
			N_X(t) &= Ne^{-\lambda_X t} \\
			N_Y(t) &= \frac{\lambda_X}{\lambda_Y - \lambda_X}N \left(e^{-\lambda_X t} - e^{-\lambda_Y t}\right)
		\end{aligned}
	\right\}.
\end{equation*}
I com que \( A_X = \lambda_X N_X \) i \( A_Y = \lambda_Y N_Y \), les activitats en funció del temps són
\begin{equation} \label{eq:activitats}
	\left.
		\begin{aligned}
			A_X(t) &= \lambda_XNe^{-\lambda_X t} \\
			A_Y(t) &= \frac{\lambda_X \lambda_Y}{\lambda_Y - \lambda_X}N \left(e^{-\lambda_X t} - e^{-\lambda_Y t}\right)
		\end{aligned}
	\right\}.
\end{equation}

Si considerem el quocient \( A_Y / A_X \) trobem
\begin{equation*}
	\frac{A_Y}{A_X} = \frac{\lambda_Y}{\lambda_Y - \lambda_X} \frac{e^{-\lambda_X t} - e^{-\lambda_Y t}}{e^{-\lambda_X t}} = \frac{\lambda_Y}{\lambda_Y - \lambda_X} \left(1 - e^{(\lambda_X - \lambda_Y) t}\right).
\end{equation*}
Per tant el quocient tindrà límit finit només quan \( \lambda_X < \lambda_Y \). Això ens indica que només tindrem equilibri quan \( \lambda_X < \lambda_Y \). En aquestes condicions \( e^{-\lambda_X t} - e^{-\lambda_Y t} \) serà semblant a \( e^{-\lambda_X t}  \) molt depressa. Així doncs, les dues activitats decauran amb la mateixa constant. Ara bé, si el temps de mitja vida de \( X \) ---que està directament relacionat amb \( \lambda_X \) com \( T_X = \log{2}/\lambda_X \)--- és comparable amb l'escala de temps que estem considerant aleshores les dues activitats decauran prou depressa.  

Considerem el cas d'equilibri secular, és a dir quan \( \lambda_X \ll \lambda_Y \). Aleshores tenim 
\begin{equation*}
	\frac{\lambda_X \lambda_Y}{\lambda_Y - \lambda_X} \approx \lambda_X.
\end{equation*}
Per tant, substituint a (\ref{eq:activitats}) tenim que 
\begin{equation*}
	A_Y(t) \approx \lambda_X N \left(e^{-\lambda_X t} - e^{-\lambda_Y t}\right) = \lambda_X N e^{-\lambda_X t} \left(1 - e^{(\lambda_X -\lambda_Y) t}\right) \approx \lambda_X N e^{-\lambda_X t}.
\end{equation*}
Per a l'última aproximació, observem que \( \lambda_X - \lambda_Y \approx -\lambda_Y \) en el règim en el que estem, i per tant el terme \( e^{(\lambda_X - \lambda_Y) t} \) decaurà ràpidament. Per tant, en l'equilibri tindrem que \( A_X = A_Y \). I a més, si el temps de mitja vida de \( X \) és molt gran comparat amb l'escala de temps que considerem, les dues activitats seran bàsicament constants un cop s'assoleixi l'equilibri.

Finalment, si \( \lambda_X > \lambda_Y \) aleshores l'activitat de \( X \) decaurà molt més depressa que la de \( Y \). Ja hem observat que en aquest cas \( A_Y/A_X \) no té límit finit, sino que tendeix a \( -\infty \), per tant no s'assolirà cap equilibri. 

Com a exemple d'equilibri transitori podem considerar la següent cadena de desintegracions:
\begin{equation*}
	{}_{42}^{99} \mathrm{Mo} \xrightarrow{\beta^-} {}_{43}^{99} \mathrm{Tc}^{\ast} \xrightarrow{\gamma} {}_{43}^{99} \mathrm{Tc}.
\end{equation*}
La desintegració \( \beta^- \) del molibdè--99 en tecneci--99 excitat té un temps de mitja vida de \SI{67}{h}, per tant una constant de desintegració de \SI{0.01}{h^{-1}}. En canvi, l'estabilització del tecneci--99 a través d'una emissió \( \gamma \) té un temps de mitja vida de \SI{6}{h} i per tant una constant de desintegració de \SI{0.12}{h^{-1}}.


\end{document}
