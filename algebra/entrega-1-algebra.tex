\documentclass[12pt]{article}

\usepackage[utf8]{inputenc}
\usepackage[T1]{fontenc}
\usepackage[catalan]{babel}
\usepackage{lmodern}
\usepackage{geometry}
\usepackage{hyperref}
\usepackage{xcolor}
\usepackage[bf,sf]{titlesec}
\usepackage{graphicx}
\usepackage{amsmath,amssymb}
\usepackage[catalan]{cleveref}

\geometry{
	a4paper,
	right = 2.5cm,
	left = 2.5cm,
	bottom = 3cm,
	top = 3cm
}
\hypersetup{
	colorlinks,
	linkcolor = {red!50!blue},
	linktoc = page
}

\renewcommand{\S}{\mathfrak{S}}
\newcommand{\A}{\mathfrak{A}}
\newcommand{\gen}[1]{\left\langle #1 \right\rangle}
\newcommand{\normal}{\trianglelefteq}
\newcommand{\parbreak}{
	\begin{center}
		--- $\ast$ ---
	\end{center} 
}
\makeatletter
\newcommand*{\defeq}{\mathrel{\rlap{%
                     \raisebox{0.3ex}{$\m@th\cdot$}}%
                     \raisebox{-0.3ex}{$\m@th\cdot$}}%
                     =}
\makeatother

\title{\sffamily {\bfseries Entrega 1:} Grups}
\author{\sffamily Arnau Mas}
\date{\sffamily 24 d'Abril 2018}


\begin{document}
\maketitle

\section*{Problema 1}
Considerem dos grups \( H \) i \( K \) amb \( e_H \) i \( e_K \) els respectius elements neutres. Aleshores el producte \( G \defeq H \times K \) té per neutre \( e = (e_H, e_K) \). Hem de veure que \( H \) és isomorf a \( H' \defeq H \times \gen{e_K} \leq G \) i que \( H' \) és normal a \( G \). Definim la següent aplicació
\begin{align*}
	\psi \colon H & \longrightarrow H \times \gen{e_K} \\
	h & \longmapsto (h, e_K). 
\end{align*}
Tenim que \( \psi \) és morfisme, ja que per tot \( h_1, h_2 \in H \) es verifica
\begin{equation*}
	\psi(h_1h_2) = (h_1h_2,e_K) = (h_1,e_K) (h_2,e_K) = \psi(h_1)\psi(h_2).
\end{equation*}
És cert que \( \psi \) és un monomorfisme, ja que \( (h_1, e_K) = (h_2,e_K) \) si i només si \( h_1 = h_2 \). I \( \psi \) també és epimorfisme ja que per tot \( g \in H' \) existeix \( h \in H \) tal que \( g = (h,e_K) \). Per tant \( \psi \) és isomorfisme. Això ens dóna \( H \cong H' \). Per veure que és normal considerem \( (x,e_K) \in H' \). Aleshores, per tot \( (h, k) \in G \) tenim
\begin{equation*}
	(h, k)^{-1}(x,e_K)(h,k) = (h^{-1}xh, k^{-1}k) = (h^{-1}xh, e_K).
\end{equation*}
I com que \( h^{-1}xh \in H \) tenim que \( (h^{-1}xh, e_K) \in H' \) i per tant concloem \( H' \normal G \). 

Tenim exactament el mateix resultat si considerem l'altre factor del producte, és a dir \( K' \defeq \gen{e_H} \times K \). Observem que tenim un isomorfisme natural de \( H \times K \) a \( K \times H \) 
\begin{align*}
	\varphi \colon H \times K &\longrightarrow K \times H \\
	(h, k) & \longmapsto (k,h).
\end{align*}
És clar que \( \varphi \) és bijectiva. I també és morfisme per com es defineix l'operació del producte de grups:
\begin{equation*}
	\varphi((h_1,k_1)(h_2,k_2)) = \varphi((h_1h_2,k_1k_2)) = (k_1k_2,h_1h_2) = (k_1,h_1)(k_2,h_2) = \varphi((h_1,k_1))\varphi((h_2,k_2)).
\end{equation*}
Així doncs tenim en particular que \( \gen{e_H} \times K \cong K \times \gen{e_H} \). Apliquem el resultat anterior a \( K \times \gen{e_H} \) dins de \( K \times H \) i tenim que és isomorf a \( K \) i normal a \( K \times H \). I per tant \( K' \) és isomorf a \( K \) i normal a \( G \).  

Finalment, comprovem que \( H' \cap K' = \gen{e} \). Efectivament, considerem \( (h,k) \in H' \cap K' \). Aleshores, com que \( (h,k) \in H' \) ha de ser \( k = e_K \). I com que \( (h,k) \in K' \) ha de ser \( h = e_H \). I per tant \( (g,h) = (e_H, e_K) = e \).

\parbreak

Hem de veure ara el recíproc al resultat previ. És a dir, si \( H,K \normal G \) i \( G = HK \) amb \( H \cap K = \gen{e} \) aleshores \( G \cong H \times K \). Com que \( G = HK \) tenim que per tot \( g \in G \) existeixen \( h\in H \) i \( k \in K \) tals que \( g= hk \). De fet, com que \( H \cap K = \gen{e} \), \( h \) i \( k \) són únics. Efectivament, si \( g = h_1k_1 = h_2k_2 \) amb \( h_1,h_2 \in H \) i \( k_1,k_2 \in K \) aleshores tenim  
\begin{equation*}
h_2^{-1}h_1 = k_2k_1^{-1}.
\end{equation*}
Com que \( h_2^{-1}h_1 \in H \) i \( k_2k_1^{-1} \in K \) això vol dir que \( h_2^{-1}h_1 = k_2k_1^{-1} \in H \cap K = \gen{e} \). Per tant \( h_1 = h_2 \) i \( k_1 = k_2 \). Això ens dóna que tot element de \( G \) s'escriu de manera única com el producte d'un element de \( H \) i un element de \( K \). Gràcies a això podem definir una aplicació 
\begin{align*}
	f \colon G & \longrightarrow H \times K \\
	g = hk & \longmapsto (h,k).
\end{align*}
Que \( f \) està ben definida ens ho dóna la unicitat de \( h \) i \( k \) que acabem de provar. Per veure que \( f \) és un morfisme ens caldrà fer servir que \( H \) i \( K \) són normals a \( G \). Considerem \( g_1 = h_1k_1 \) i \( g_2 = h_2k_2 \) amb \( h_1,h_2 \in H \) i \( k_1,k_2 \in K \). Per veure que \( f \) és morfisme hem de provar que \( g_1g_2 = h_1h_2k_1k_2 \) ja que \( f(g_1)f(g_2) = (h_1,k_1)(h_2,k_2) = (h_1h_2,k_1k_2) \). Tenim que \( g_1g_2 = h_1k_1h_2k_2 \). Com que \( H \) és normal, existeix \( h_3 \in H \) tal que \( k_1h_2 = h_3k_1 \). Similarment, per la normalitat de \( K \) existeix \( k_3 \in K \) tal que \( k_1h_2 = h_2k_3 \). Això ens dóna \( g_1g_2 = h_1h_2k_3k_2 = h_1h_3k_1k_2 \). Però pel que hem provat prèviament ha de ser \( h_1h_2 = h_1h_3 \) i \( k_1k_2 = k_3k_2 \). Per tant \( f(g_1g_2) = f(h_1h_2k_1k_2) = (h_1h_2,k_1k_2) = f(g_1)f(g_2) \) i \( f \) és morfisme. \( f \) és epimorfisme ja que per tot \( (h,k) \in H \times K \) tenim \( f(hk) = (h,k) \). I també és monomorfisme ja que \( ee = e \). Així doncs tenim \( G = HK \cong H \times K \) quan \( H \) i \( K \) són subgrups normals amb intersecció trivial.

\parbreak

A continuació generalitzem el resultat anterior per a qualsevol nombre de subgrups. És a dir, considerem un grup \( G \) amb \( H_1 \dots H_n \) subrups normals a \( G \) i 
\begin{equation*}
	H_i \cap \left(H_1 \dots H_{i-1} H_{i+1} \dots H_n\right) 
\end{equation*}
per tot \( i \in \{1, \cdots, n\} \) (podem definir \( H_0 = H_{n+1} = \gen{e} \) per evitar problemes amb el rang de \( i \)). Aleshores si \( G = H_1\cdots H_n \) es té \( G \cong H_1 \times \cdots \times H_n \).

Procedim per inducció sobre \( n \). El cas \( n = 2 \) és l'apartat anterior. Considerem, per tot \( n \in \mathbb{N} \), \( G = H_1 \cdots H_{n+1} \) amb \( H_1, \cdots , H_{n+1} \) subgrups en les condicions anteriors. Com que el producte de sugrups normals és normal, \( H_1\cdots H_n \) és normal a \( G \). A més \( \left(H_1 \cdots H_n \right) \cap H_{n+1} = \gen{e} \), per tant podem aplicar l'apartat anterior per obtenir \( G \cong \left(H_1 \cdots H_n\right) \times H_{n+1} \). I si apliquem la hipòtesi d'inducció a \( H_1 \cdots H_n \) trobem 
\begin{equation*}
	G \cong \left(H_1 \times \cdots \times H_n\right) \times H_{n+1} \cong H_1 \times \cdots \times H_{n+1}.
\end{equation*}

\pagebreak
\section*{Problema 2}


\end{document}
