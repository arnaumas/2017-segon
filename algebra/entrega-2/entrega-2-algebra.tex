\documentclass[12pt]{article}

\usepackage[utf8]{inputenc}
\usepackage[T1]{fontenc}
\usepackage[catalan]{babel}
\usepackage{lmodern}
\usepackage{geometry}
\usepackage{hyperref}
\usepackage{xcolor}
\usepackage{enumitem}
\usepackage[bf,sf,small,pagestyles]{titlesec}
\usepackage{graphicx}
\usepackage{amsmath,amssymb}
\usepackage[catalan,sort]{cleveref}

% Marges
\geometry{
	a4paper,
	right = 2.5cm,
	left = 2.5cm,
	bottom = 3cm,
	top = 3cm
}

% Referències
\hypersetup{
	colorlinks,
	linkcolor = {red!50!blue},
	linktoc = page
}

% Instruccions noves
\newcommand{\Z}{\mathbb{Z}}
\newcommand{\R}{\Z[\sqrt{-3}]}
\newcommand{\N}{\mathbb{N}}
\newcommand{\C}{\mathbb{C}}
\newcommand{\unit}[1]{#1^{\times}}
\newcommand{\gen}[1]{\langle #1 \rangle}
\newcommand{\abs}[1]{\left\lvert #1 \right\rvert}
\newcommand{\normal}{\trianglelefteq}
\newcommand{\id}{\mathrm{id}}
\newcommand{\parbreak}{
	\begin{center}
		--- $\ast$ ---
	\end{center} 
}
\makeatletter
\newcommand*{\defeq}{\mathrel{\rlap{%
    \raisebox{0.3ex}{$\m@th\cdot$}}%
  \raisebox{-0.3ex}{$\m@th\cdot$}}%
	=
}
\makeatother

% Estil de pàgina
\newpagestyle{pagina}{
	\headrule
	\sethead*{}{}{\sffamily \sectiontitle}
	\footrule
	\setfoot*{}{}{\sffamily \thepage}
}
\renewpagestyle{plain}{
	\footrule
	\setfoot*{}{}{\sffamily \thepage}
}
\pagestyle{pagina}

\titleformat{\section}[hang]{\bfseries \sffamily \Large}{}{0pt}{}{\thispagestyle{plain}}

% Títol i autor
\title{\sffamily {\bfseries Entrega 2:} Divisibilitat i dominis euclidians}
\author{\sffamily Arnau Mas}
\date{\sffamily 18 de maig 2018}

\begin{document}
\maketitle

Considerem l'anell \( \R \defeq \Z + \sqrt{-3}\,\Z \subseteq \C \) amb la suma i producte de \( \C \). Hi podem definir l'aplicació
\begin{align*}
	N \colon \R & \longrightarrow \N \\
	z = a + b\sqrt{-3} & \longmapsto a^2 + 3b^2.
\end{align*}

Comprovem que \( N \) és una norma, és a dir, que és definida estrictament positiva i que és multiplicativa. És clar que per tot \( z \in \R \) es té \( N(z) \geq 0 \). També és clar que \( N(0) = 0 \). Finalment, si \( N(z) = 0 \) hem de poder concloure \( z = 0 \). Si \( z = a + b \sqrt{-3} \in \R \) i \( N(z) = 0 \) tenim \( a^2 + 3b^2 = 0 \), però com que \( a^2, b^2 > 0 \) si \( a,b > 0 \) ha de ser \( a = b = 0 \) i per tant \( z = 0 \).

Per veure que \( N \) és multiplicativa observem que \( N(z) = z\bar{z} \) per tot \( z \in \R \). Efectivament, si \( z = a+b\sqrt{-3} \) tenim
\begin{equation*}
	z\bar{z} = (a + \sqrt{-3}b)(a - \sqrt{-3}b) = a^2 + 3b^2 = N(z).
\end{equation*}
Per tant, per tot \( z, w \in \R \) tenim
\begin{equation*}
	N(zw) = zw \bar{zw} = z\bar{z} w\bar{w} = N(z) N(w).
\end{equation*}

\parbreak

Considerem \( u \in \R \) una unitat. Per tant existeix \( u^{-1} \in \R \) tal que \( uu^{-1} = 1 \). Fent servir que la norma és multiplicativa i que \( N(1) = 1 \) tenim
\begin{equation*}
	1 = N(1) = N(uu^{-1}) = N(u) N(u^{-1}).
\end{equation*}
Per tant ha de ser \( N(u) = 1 \) o \( N(u) = -1 \) ja que \( u \neq 0 \). Però com que \( N(u) > 0 \) concloem \( N(u) = 1 \).  

I si \( z \in \R \) compleix \( N(z) = 1 \) tenim \( z\bar{z} = 1 \) i per tant \( z \) és una unitat amb \( z^{-1} = \bar{z} \).

Per determinar \( \unit{\R} \) només cal que determinem tots els \( z \in \R \) tals que \( N(z) = 1 \). És a dir, hem de trobar totes les solucions de \( a^2 + 3b^2 = 1 \) amb \( a, b \in \Z \). Observem que ha de ser \( b = 0 \) ja que si \( b \neq 0 \) aleshores \( b^2 \geq 1 \) i per tant \( a^2 + 3b^2 \geq 3 \). Per tant les úniques possibles opcions són \( a = 1 \) o \( a = -1 \). Així doncs les unitats de \( \R \) són
\begin{equation*}
	\unit{\R} = \{1, -1\}.
\end{equation*}

\parbreak

Considerem \( z \in \R \) amb \( N(z) = 4 \). Observem primer que \( z \neq 0 \) ja que \( N(z) \neq 0 \). Si existeixen \( \alpha, \beta \in \R \) tals que \( z = \alpha \beta \) aleshores hem de tenir \( N(\alpha) N(\beta) = N(\alpha\beta) = 4 \). És a dir, \( N(\alpha), N(\beta) \mid 4 \). Per tant els únics valors possibles per \( N(\alpha) \) i \( N(\beta) \) són 1, 2 o 4. Els valors que pot pendre la norma, però, estan molt restringits per a valors petits. De fet, la norma d'un element no pot ser mai 2. En efecte, si \( z = a + \sqrt{-3}b \in \R \) complís \( N(z) = 2 \) tindriem \( a^2 + 3b^2 = 2 \). Però això és impossible ja que si \( b \neq 0 \) aleshores \( b^2 > 1 \) i per tant \( N(z) = a^2 + 3b^2 \geq 3 \). I si \( b = 0 \) aleshores \( a^2 \) no pot ser 2 si \( a \in \Z \). 

El que tenim és que o bé \( N(\alpha) = 1 \) o bé \( N(\alpha) = 4 \). En el primer cas \( \alpha \) és una unitat, per l'apartat anterior. I en el segon ha de ser \( N(\beta) = 1 \) i per tant ara és \( \beta \) la que és una unitat. Sigui com sigui, acabem de provar que si \( z \) descomposa en producte de dos factors, almenys un dels dos sempre és una unitat, i per tant \( z \) és irreductible.    

\parbreak

Considerem \( z = 2 \) i \( w = 1 + \sqrt{-3} \). Tenim \( N(z) = N(w) = 4 \) i per tant, per l'apartat anterior, els dos són irreductibles.  

Volem provar que 1 és un màxim comú divisor de \( z \) i \( w \). És clar que \( z \) i \( w \) no són associats ja que les úniques unitats de \( \R \) són 1 i \( -1 \). També és evident que 1 és divisor comú de \( z \) i \( w \). Hem de veure que qualsevol altre divisor comú de \( z \) i \( w \) és una unitat. Considerem, doncs, \( d \in \R \) un divisor comú de \( z \) i \( w \). És a dir, podem escriure \( z = \alpha d \) i \( w = \beta d \) per alguns \( \alpha, \beta \in \R \). Com que \( z \) i \( w \) són irreductibles hi ha dues possibilitats: o bé \( d \) és una unitat o bé \( \alpha \) i \( \beta \) són unitats. En el primer cas ja hem acabat. I si el segon fos cert tindriem que \( z \) i \( w \) són associats, però això sabem que no és veritat. Per tant \( d \) ha de ser una unitat i concloem que 1 és un màxim comú divisor de \( z \) i \( w \).    

Si \( z \) i \( w \) compleixen una identitat de Bézout vol dir que existeixen \( \lambda, \mu \in \R \) tals que 
\begin{equation*}
	\lambda z + \mu w = 1.
\end{equation*}
Posem \( \lambda = \lambda_1 + \lambda_2\sqrt{-3} \) i \( \mu = \mu_1 + \mu_2 \sqrt{-3} \). Estem dient, doncs, que 
\begin{equation*}
	2\lambda_1 + \mu_1 -3\mu_2 + (2 \lambda_2 + \mu_1 + \mu_2)\sqrt{-3} = 1.
\end{equation*}
Per tant ha de ser \( 2\lambda_2 + \mu_1 + \mu_2 = 0 \) i \( 2\lambda_1 + \mu_1 - 3\mu_2 = 1 \). Restant trobem
\begin{equation*}
	2(\lambda_1 - \lambda_2) - 4\mu_2 = 1. 
\end{equation*}
Però aleshores \( \lambda_1, \mu_1 \) i \( \mu_2  \) no poden existir ja que tant \( 2(\lambda_1 - \lambda_2) \) com \( 4\mu_2 \) serien parells i \( -1 \) no és parell. Per tant \( z \) i \( w \) no satisfan una identitat de Bézout.  

\parbreak

Tenim que \( zw = 2(1 + \sqrt{-3}) = 2 + 2 \sqrt{-3} \) és un múltiple comú de \( z \) i \( w \). També ho és \( z^2 = w \bar{w} = 4 \) i \( N(z^2) = N(w \bar{w}) = 16 \). Si \( z \) i \( w \) tinguessin un mínim comú múltiple \( m \) hauria de passar, per definició, \( m \mid 4 \) i \( m \mid (2 + 2\sqrt{-3}) \). Això ens diu \( N(m) \mid 16 \). Com hem comentat abans, la norma d'un element de \( \R \) no pot ser 2, per tant \( N(m) \neq 8 \) ja que si fos així i escribim \( 4 = am \) aleshores \( N(a) = 2 \): una contradicció. A més, com que, per definició, \( z \mid m \) i \( w \mid m \) hem de tenir \( N(m) \geq 4 \). Les úniques opcions possibles, doncs, són \( N(m) = 16 \) i \( N(m) = 4 \).

Si \( N(m) \) fos 16 aleshores tindriem que \( m \) és associat tant a \( z^2 \) com a \( w\bar{w} \) ---com que \( m \mid z^2 = 4 \) tenim \( m = 4b \) per \( b \in \R \), però aleshores \( N(b) = 1 \) i \( b \) és una unitat; i de la mateixa manera comprovem que \( m \) i \( w\bar{w} \) serien associats---, però això sabem que no és el cas ja que \( w\bar{w} \) i \( z^2 \) no són associats.  

I si \( N(m) \) fos 4 aleshores, per un apartat anterior, \( m \) seria irreductible; en contradicció amb la hipotesi que \( z \mid m \) i \( w \mid m \) ---i amb que \( z \) i \( w \) no són associats. Així doncs concloem que \( m \) no pot existir i que \( z \) i \( w \) no tenen mínim comú múltiple.   

\parbreak

Hem de veure que \( 2w \) i \( 2z \) no tenen màxim comú divisor. Tenim \( 2w = zw = 2 + 2\sqrt{-3} \) i \( 2z = z^2 = w\bar{w} = 4 \). És clar que \( z = 2 \) és un divisor comú de \( 2z \) i \( 2w \), així com \( w \). Així, si \( d \) és un màxim comú divisor de \( 2w \) i \( 2z \) hem de tenir \( z \mid d \) i \( w \mid d \). A més \( d \mid 2w \) i \( d \mid 2z \). Això ens restringeix molt \( N(d) \): concretament, \( 4 \mid N(d) \) i \( N(d) \mid 16 \). Seguint el mateix argument que a l'apartat anterior, no és possible que \( N(d) = 8 \). Si \( N(d) = 4 \) és irreductible, la qual cosa implicaria que tant \( z \) com \( w \) són associats a \( d \) i per tant associats entre si, però sabem que no és el cas. I si \( N(d) = 16 \) podriem concloure que \( 2w \) i \( 2z \) són associats, que també sabem que no és el cas. Així doncs, \( d \) no pot existir i per tant \( 2z \) i \( 2w \) no tenen màxim comú divisor.   



\parbreak

Tenim 
\begin{equation*}
	4 = 3 \cdot 2 = (1 + \sqrt{-3})(1 - \sqrt{-3}).
\end{equation*}
Ja hem vist que tant \( 2 \) com \( 1 + \sqrt{-3} \) són irreductibles no associats. També és irreductible \( 1 - \sqrt{-3} \) ja que \( N(1 - \sqrt{-3}) = 4 \). I 2 tampox no és associat a \( 1 - \sqrt{-3} \). Per tant acabem d'escriure 4 com a producte d'irreductibles de dues maneres diferents. Això ens dóna que \( \R \) no és un DFU.    


\end{document}
