\section*{Problema 3}
Sen's demana calcular el volum de la intersecció entre l'interior de l'el·lipsoide d'equació \[ \dfrac{x^2}{a^2} + \dfrac{y^2}{b^2} + \dfrac{z^2}{c^2} = 1 \] i el semiespai donat per \[ Ax + By + Cz \leq D . \] Observem que podem fer un canvi de coordenades per a reduir al problema al càlcul del volum d'un casquet esfèric. Concretament el canvi que fem servir és el definit per 
\begin{align*}
	H \colon \R^3 & \longrightarrow \R^3 \\
	(x,y,z) & \longmapsto \left(\dfrac{x}{a},\dfrac{y}{b},\dfrac{z}{c}\right).
\end{align*}
Si denotem les noves coordenades per \( (u,v,w) \) tenim que ara l'el·lipsoide té equació \( u^2 + v^2 + w^2 = 1 \) i ja no és un el·lipsoide sino que és una esfera. L'equació del semiespai es transforma en \( Aau + Bbu + Ccw \leq D \). És clar que \( H \) és una bijecció lineal i per tant un difeomorfisme. Calcular-ne la inversa és immediat i tenim que el jacobiá de \( H^{-1} \) és \( J_{H^{-1}}=abc \). Així doncs, pel teorema del canvi de variable tenim que si denotem per \( A \) el conjunt del qual en volem calcular el volum:
\begin{equation}
	m(A) = \int_A \mathbf{1}_A = \int_{H(A)} \mathbf{1}_{H(A)} \abs{J_{H^{-1}}} = \abs{abc} \int_{H(A)} \mathbf{1}_{H(A)} = \abs{abc}m(H(A)). \label{eq:canvi de variables per simplificar}
\end{equation}
Podem suposar, sense pèrdua de generalitat, que \( a \), \( b \) i \( c \) són reals positius.

La regió de la qual hem de calcular el volum ara, \( H(A) \) és un casquet esfèric. El volum, \( V \), d'un casquet esfèric d'una esfera de radi \( r \) està donat per
\begin{equation}
	V = \dfrac{\pi h^2}{3}(3r - h), \label{eq:volum d'un casquet}
\end{equation}
on \( h \) és l'alçada del casquet. En el nostre cas, \( r = 1 \). Ens serà útil reescriure \ref{eq:volum d'un casquet} en funció de la distància entre el pla que defineix el casquet i l'origen. La distància, \( d \), entre el pla ---després del canvi de coordenades---	i l'origen ve donada per 
\begin{equation}
	d = \dfrac{D}{\sqrt{A^2a^2 + B^2b^2 + C^2c^2}}. \label{eq:definicio de d}
\end{equation}
Tal i com la definim, \( d \) pot ser negativa o positiva en funció del signe de \( D \). Això distingeix si el pla es troba per sota de l'origen, \( d < 0 \), o per sobre, \( d > 0 \). Tenim que \( h \) serà més petita que \( 1 \) quan \( d < 0 \), i de fet \( h - d = 1 \). Així doncs, substituïnt a \ref{eq:volum d'un casquet} obtenim
\begin{equation}
	m(H(A)) = \dfrac{\pi(1 + d)^2}{3}(2 - d) = \dfrac{\pi}{3}\left(2 + 3d - d^3\right).
\end{equation}
I finalment, recuperant \ref{eq:canvi de variables per simplificar} tenim
\begin{equation}
	m(A) = \dfrac{\pi}{3}abc (2 + 3d - d^3)
\end{equation}
amb \( d \) definida com a \ref{eq:definicio de d}. Ometem el valor absolut ja que, tal i com hem indicat abans, podem suposar sense pèrdua de generalitat que \( a, b, c \geq 0 \).




