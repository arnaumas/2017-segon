\documentclass[12pt]{article}

\usepackage[utf8]{inputenc}
\usepackage[T1]{fontenc}
\usepackage[catalan]{babel}
\usepackage{lmodern}
\usepackage{geometry}
\usepackage{hyperref}
\usepackage{xcolor}
\usepackage[bf,sf,small,pagestyles]{titlesec}
\usepackage[font={footnotesize, sf}, labelfont=bf]{caption} 
\usepackage{siunitx}
\usepackage{graphicx}
\usepackage{booktabs}
\usepackage{amsmath,amssymb}
\usepackage[catalan,sort]{cleveref}

\geometry{
	a4paper,
	right = 2.5cm,
	left = 2.5cm,
	bottom = 3cm,
	top = 3cm
}

\hypersetup{
	colorlinks,
	linkcolor = {red!50!blue},
	linktoc = page
}

\crefname{figure}{figura}{figures}
\crefname{table}{taula}{taules}
\numberwithin{table}{section}
\numberwithin{figure}{section}
\numberwithin{equation}{section}

\graphicspath{{./figs/}}

% Unitats
\sisetup{
	inter-unit-product = \ensuremath{ \cdot },
	allow-number-unit-breaks = true,
	detect-family = true,
	list-final-separator = { i },
	list-units = single
}

\newcommand{\Z}{\mathbb{Z}}
\newcommand{\N}{\mathbb{N}}
\newcommand{\R}{\mathbb{R}}
\DeclareMathOperator{\gr}{gr}
\newcommand{\abs}[1]{\left\lvert #1 \right\rvert}
\newcommand{\inn}[2]{\left\langle #1 , #2 \right\rangle}
\newcommand{\parbreak}{
	\begin{center}
		--- $\ast$ ---
	\end{center} 
}
\makeatletter
\newcommand*{\defeq}{\mathrel{\rlap{%
    \raisebox{0.3ex}{$\m@th\cdot$}}%
  \raisebox{-0.3ex}{$\m@th\cdot$}}%
=}
\makeatother

\newpagestyle{pagina}{
	\headrule
	\sethead*{\sffamily {\bfseries Pràctica 4:} Quadratura de Gauss}{}{\sffamily Arnau Mas}
	\footrule
	\setfoot*{}{}{\sffamily \thepage}
}
\renewpagestyle{plain}{
	\footrule
	\setfoot*{}{}{\sffamily \thepage}
}
\pagestyle{pagina}

\title{\sffamily {\bfseries Pràctica 4:} Quadratura de Gauss}
\author{\sffamily Arnau Mas}
\date{\sffamily 6 de juny 2018}

\begin{document}
\maketitle

\section{Introducció}
Les fórmules de quadratura gaussianes apareixen per intentar millorar la precisió de les regles de quadratura de Newton-Cotes. Fent una tria precisa dels nodes es pot reduir molt la fita de l'error comès. Concretament, posem que tenim unz funció \( f \colon [a,b] \longrightarrow \R \) integrable i \( \omega \colon [a,b] \longrightarrow [0, \infty) \) una funció pes no negativa en \( [a,b] \). Aleshores aproximarem la integral \( \int_a^b \omega(x)f(x) \, dx \) per
\begin{equation*}
 	\int_a^b \omega(x)f(x) \, dx \approx \sum_{i = 1}^n \omega_i f(\alpha_i). 
\end{equation*}
Els \( \alpha_i \) s'anomenen els nodes i els \( \omega_i \) reben el nom de pesos. L'utilitat de la quadratura de Gauss apareix per a una bona tria de nodes ---un cop triats els nodes els pesos queden determinats imposant exactitud de la fórmula per a polinomis de graus entre 0 i \( n-1 \).  

La funció pes \( \omega \) indueix un producte interior semidefinit positiu a l'espai de funcions definit com
\begin{equation*}
	\inn{f}{g} \defeq \int_a^b \omega(x) f(x) g(x) \, dx.
\end{equation*}
Es diu que és semidefinit positiu perquè per tota \( f \) integrable no nu\l.la \( \inn{f}{f} \geq 0 \). Aleshores podem definir una noció d'ortogonalitat dient que \( f \) i \( g \) són ortogonals si i només si \( \inn{f}{g} = 0 \). Direm que una família de polinomis \( (p_n) \) és ortogonal si \( \inn{p_i}{p_j} = 0 \) per tot \( i \neq j \). Es pot demostrar que per tot pes existeix una única família ortogonal \( (p_n) \) de polinomis mònics i tals que \( \gr p_n = n \). A més, cada membre de la família \( p_n \) té \( n \) arrels diferents totes contingudes a \( [a,b] \). Precisament aquestes arrels són els nodes de la corresponent fórmula de quadratura de gauss. Els dos pesos que farem servir en aquesta pràctica són \( \omega(x) = 1 \), que dóna lloc als polinomis de Legendre, i \( \omega(x) = (1 - x^2)^{-1/2} \), que dóna lloc als polinomis de Chebyshev ---ambdós pesos estan definits a \( [-1,1] \). A partir d'ara \( L_n \) denotarà el polinomi de Legendre de grau \( n \) i \( C_n \) denotarà el polinomi de Chebyshev de grau \( n \). 

\section{Càlcul dels nodes}

Existeixen fórmules recursives per a trobar els polinomis de Legendre i de Chebyshev, concretament
\begin{equation} \label{eq:recursio legendre}
	L_n = \frac{2n-1}{n}xL_{n-1} - \frac{n-1}{n}L_{n-2}
\end{equation}
i
\begin{equation} \label{eq:recursio chebyshev}
	C_n = 2xC_{n-1} - C_{n-2},
\end{equation}
amb \( L_0 = C_0 = 1 \) i \( L_1 = C_1 = x \). 

Al programa \texttt{polinomis.c} hi ha rutines que implementen les \cref{eq:recursio legendre,eq:recursio chebyshev} per calcular els polinomis de Chebyshev i Legendre de grau \( n \). També hi ha implementada una rutina que detecta els punts on un polinomi canvia de signe avaluant-lo a increments petits. Com que sabem que \( L_n \) i \( C_n \) tenen \( n \) arrels diferents a \( [-1,1] \), aquesta rutina ens permet trobar una primera aproximació d'aquestes: guarda el punt mig entre dos punts on el polinomi té signe diferent, sabent que ha de trobar exactament \( n \) canvis de signe. Si no els troba, ho torna a repetir avaluant a increments més petits. Seguidament, aquestes primeres aproximacions es milloren fent servir el mètode de Newton amb una tolerància donada. Amb tot això tenim els nodes per a les quadratures de Gauss-Legendre i Gauss-Chebyshev.     

\section{Càlcul dels pesos}

Ja hem mencionat que els pesos \( \omega_i \) queden determinats imposant exactitud de la fórmula. Concretament imposem
\begin{equation*}
	\int_{-1}^{1}\omega(x) x^k \,dx = \sum_{i = 1}^n \omega_i \alpha_i^k
\end{equation*}
per tot \( 0 \leq k \leq n-1 \). Fem el càlcul explícit per les quadratures de Gauss-Legendre i Gauss-Chebyshev. Pel cas de Legendre es té \( \omega(x) = 1 \) i 
\begin{equation*}
	\int_{-1}^{1}x^k\,dx = \left[\frac{x^{k+1}}{k+1}\right]^{1}_{-1} = \frac{1}{k+1}\left(1 + (-1)^k\right). 
\end{equation*}
Per tant, per tot \( 0 \leq k \leq n-1 \) s'ha de verificar 
\begin{equation*}
	\sum_{i = 1}^n \omega_i \alpha_i^k = \frac{1}{k+1}\left(1 + (-1)^k\right). 
\end{equation*}
Podem escriure-ho en forma matricial com
\begin{equation*}
asfdjk
\end{equation*}


\end{document}
