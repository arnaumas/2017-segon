\documentclass[12pt]{article}
\usepackage{header}

\title{Lliurament de Càlcul de Diverses Variables i Optimització}
\author{Arnau Mas}
\date{26 d'Octubre de 2017}

\begin{document}
\maketitle

\section*{Problema 3}
Per veure que \( q \) és una norma observem el següent:

\begin{IEEEeqnarray*}{rCl}
  q(x,y) &=& \sqrt{(x+2y)^{2} + (y-2x)^{2}} \\
  			 &=& \sqrt{5x^{2} + 5y^{2}} \\
				 &=& \norm{5(x,y)}
\end{IEEEeqnarray*}
on \( \norm{\cdot}\) és la norma euclidiana. Veiem, doncs, que \( q \) prové d'un producte escalar. Explícitament 
\[ q(x) = \sqrt{\inn{5x}{5x}}. \]
Per tant ara és clar que \( q \) és una norma.

\bigskip
Considerem \( T \colon \R^{2} \to \R^{2} \) tal que, per tot \( x, y \in \R^{2} \), es té \( q(T(x) - T(y)) = q(x,y) \). Pel que hem vist abans, això equival a dir que \( \sqrt{5}\norm{T(x) - T(y)} = \sqrt{5}\norm{x-y} \), que és a la vegada equivalent a dir que \( T \) és una isometria, és a dir, que conserva la distància euclidiana. Per tant, hem de veure que qualsevol isometria  que fixa l'origen és lineal. Com a observació prèvia veiem que una isometria conserva la norma: 
\begin{equation*}
	\norm{T(x)} = \norm{T(x) - T(0)}  = \norm{x-0} = \norm{x}
\end{equation*}
No només això, sino que també conserva el producte escalar. En efecte, tenim que \( \norm{x - y}^2 = \norm{x}^2 + \norm{y}^2 - 2\inn{x}{y} \). Per tant
\begin{equation*}
   \inn{x}{y} = \dfrac{1}{2} \left( \norm{x}^2 + \norm{y}^2 - \norm{x - y}^2\right)
\end{equation*}
Ara és immediat veure que \( T \) conserva el producte escalar:
\begin{IEEEeqnarray*}{rCl}
	\inn{T(x)}{T(y)} & = & \dfrac{1}{2} \left( \norm{T(x)}^2 + \norm{T(y)}^2 - \norm{T(x) - T(y)}^2\right) \\
	 								 & = & \dfrac{1}{2} \left( \norm{x}^2 + \norm{y}^2 - \norm{x - y}^2 \right) \\ 
									 & = & \inn{x}{y} 
\end{IEEEeqnarray*}

Per provar que \( T \) és lineal n'hi ha prou amb que vegem que per tot \( x, y \in \R^{2} \) i \( \lambda, \mu \in \R \) és té \( T(\lambda x + \mu y) = \lambda T(x) + \mu T(y) \). Però això és equivalent a dir que \( \norm{T(\lambda x + \mu y) - \lambda T(x) - \mu T(y)} = 0 \). Calculem, doncs, el quadrat de la norma d'aquest vector, que serà zero si i només si la pròpia norma és zero. Ho fem així per utilitzar la bilinealitat del producte escalar: 
\begin{IEEEeqnarray*}{rCl}
	\IEEEeqnarraymulticol{3}{l}{\norm{ T(\lambda x + \mu y) - \lambda T(x) - \mu T(y) }^2 } \\
  & = & \norm{ T(\lambda x + \mu y) }^2 + \norm{ \lambda T(x) + \mu T(y)}^2 - 2\inn{T(\lambda x + \mu y)}{\lambda T(x) + \mu T(y)} \\
	& = & \norm{T(\lambda x + \mu y)}^2 + \lambda^2 \norm{T(x)}^2 + \mu^2 \norm{T(y)}^2 + 2\lambda\mu \inn{T(x)}{T(y)} \\
	& & \negmedspace {}	- 2 \lambda\inn{T(\lambda x + \mu y)}{T(x)} -2 \mu\inn{T(\lambda x + \mu y)}{T(y)} \\
  & = & \norm{\lambda x + \mu y}^2 + \lambda^2 \norm{x}^2 + \mu^2 \norm{y}^2 + 2\lambda\mu\inn{x}{y} \\ 
	& & \negmedspace {} - 2\lambda\inn{\lambda x + \mu y}{x} - 2\mu\inn{\lambda x + \mu y}{y} \\ 
  & = & 2 \norm{\lambda x + \mu y}^2 - 2 \left(\lambda^2 \norm{x}^2 + \mu^2 \norm{y}^2 + 2\lambda\mu\inn{x}{y}\right) \\ 
  & = & 0
\end{IEEEeqnarray*}

\bigskip
Considerem ara una funció \( f \colon \R^2 \to \R^2 \) que compleix que existeix \( C > 0 \) tal que , per tot \( x, y \in \R^d \) es compleix
\[ \norm{f(x) - f(y)} \leq C\norm{x - y}^2. \]
Hem de veure que, en aquestes condicions, \( f \) és constant. Com que ens serà útil per al següent apartat, veurem un resultat més general. I és que només ens cal que es compleixi que 
\[ \norm{f(x) - f(y)} \leq C\norm{x - y}^{1 + \beta} \]
per algun \( \beta > 0 \) per poder afirmar que \( f \) és constant. Que sigui certa aquesta desigualtat implica que la diferencial de \( f \) és nul·la en qualsevol punt, i per tant que \( f \) és constant ---ja que \( \R^2 \) és òbviament un obert arc-connex---. Verifiquem explícitament que en tot punt \( a \in \R^2 \) és té \( df(a) = 0 \). Segons la definició de diferencial, si és veritat que \( f \) té diferencial nul·la a tot arreu, hem de tenir, per tot \( a \in \R^2 \):
\begin{equation*}
  f(a + h) = f(a) + o(\norm{h}),
\end{equation*}
o el que és el mateix
\begin{equation*}
  \lim_{\norm{h} \to 0}{\dfrac{\norm{f(a+h) - f(a)}}{\norm{h}}} = 0.
\end{equation*}
Per hipòtesi, tenim que \( 0 \leq \norm{f(a + h) -f(a)} \leq C \norm{h}^{1 + \beta} \). A més, \( C \norm{h}^{1+\beta} \to 0 \) quan \( \norm{h} \to 0 \). Per tant,
\[ \lim_{\norm{h} \to 0}{\norm{f(a + h) - f(a)}} = \lim_{\norm{h} \to 0}{C \norm{h}^{1 + \beta}} = 0. \]
I aleshores el límit que volíem avaluar es redueix a
\[ \lim_{\norm{h} \to 0}{\dfrac{C \norm{h}^{1 + \beta}}{\norm{h}}} = \lim_{\norm{h} \to 0}{C \norm{h}^{\beta}} = 0. \]
Per tant, efectivament, \( f \) té diferencial nul·la a tot arreu i per tant és constant.

\bigskip
Pel que fa a l'últim apartat, podem considerar tres casos: \( \alpha > 1 \), \( 0 < \alpha < 1 \) i \( \alpha < 0 \)
Si \( \alpha = 0 \), aleshores no pot existir \( f \colon \R^2 \to \R^2 \) bijectiva que compleixi la condició. En efecte, si \( f \) és bijectiva en particular és exhaustiva. Per tant el 0 té preimatge. Però aleshores tenim que la norma de qualsevol imatge és 1, i per tant qualsevol punt amb norma diferent de 1 no té preimatge: una contradicció.

El cas \( \alpha > 1 \) ja l'hem resolt a l'apartat anterior. Si \( \alpha > 1 \), podem posar \( \alpha = 1 + \beta \) amb \( \beta > 0 \). Però aleshores tenim, per tot \( x, y \in \R^2 \)
\[ \norm{f(x) - f(y)} = \norm{x - y}^{1 + \beta} \leq \norm{x - y}^{1 + (1+\beta)} \]
i per tant \( f \) ha de ser constant, per la qual cosa no pot ser bijectiva. 

Suposem ara que \( f \) és bijectiva i compleix la condició de l'enunciat amb \( 0 < \alpha < 1 \). Aleshores \( f \) té una inversa, també bijectiva, i tenim, per tot \( x, y \in \R^2 \)
\[ \norm{x - y} = \norm{f^{-1}(x) - f^{-1}(y)}^{\alpha} \]
i per tant
\[ \norm{x - y}^{1 / \alpha} = \norm{f^{-1}(x) - f^{-1}(y)}. \]
Però \( \dfrac{1}{\alpha} > 1 \), i per tant, tal i com hem raonat abans, \( f^{-1} \) ha de ser constant: una contradicció.

\end{document}
